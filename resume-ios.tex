\documentclass[10pt, a4paper]{awesome-cv}

% 编码设置
\usepackage[UTF8]{ctex}
\usepackage{fontspec}
\usepackage{xeCJK}

% 设置中文字体
\setCJKmainfont{PingFang SC}
\setCJKsansfont{PingFang SC}
\setCJKmonofont{PingFang SC}

% 个人信息配置
\name{张}{三}
\address{北京市朝阳区}
\mobile{138-8888-8888}
\email{zhangsan@email.com}
\github{zhangsan}
\linkedin{zhangsan}
\homepage{zhangsan.dev}

% App Store 链接(可选)
\extrainfo{\faAppStore\ https://apps.apple.com/developer/zhangsan}

% 简历主题颜色(iOS 蓝色)
\colorlet{awesome}{awesome-blue}

% 开始文档
\begin{document}

% 标题部分
\makecvheader

% 个人简介部分
\cvsection{个人简介}
\begin{cvparagraph}
资深 iOS 开发工程师,具有 6 年移动应用开发经验。精通 Swift 和 Objective-C,主导开发过多款下载量超百万的应用。熟练掌握 iOS 全栈开发,包括 UI/UX 设计、性能优化、架构设计和团队管理。对 Apple 生态系统有深入理解,热衷于探索最新的 iOS 技术和最佳实践。
\end{cvparagraph}

% 工作经历部分
\cvsection{工作经历}

\cventry
  {2021年3月 - 至今}
  {高级 iOS 工程师}
  {移动互联网科技公司}
  {北京}
  {
    \begin{cvitems}
      \item {主导公司核心产品 iOS 应用开发,使用 Swift 和 SwiftUI 构建现代化用户界面}
      \item {实现 MVVM + Combine 架构模式,提升代码可维护性和团队协作效率}
      \item {优化应用性能,启动时间减少 70\%,内存使用降低 40\%}
      \item {集成 Core Data、CloudKit 实现数据同步和离线功能}
      \item {负责 App Store 上架和版本管理,保持 4.8+ 评分}
    \end{cvitems}
  }

\cventry
  {2019年6月 - 2021年2月}
  {iOS 开发工程师}
  {金融科技公司}
  {上海}
  {
    \begin{cvitems}
      \item {开发金融类 iOS 应用,涉及支付、理财等核心功能}
      \item {使用 Face ID 和 Touch ID 实现生物识别安全认证}
      \item {集成 Apple Pay 和银联支付 SDK}
      \item {实现推送通知和消息中心功能}
      \item {参与应用安全审计和代码审查}
    \end{cvitems}
  }

\cventry
  {2018年7月 - 2019年5月}
  {初级 iOS 开发工程师}
  {创业公司}
  {深圳}
  {
    \begin{cvitems}
      \item {负责公司首款 iOS 应用的开发和维护}
      \item {使用 Objective-C 进行现有功能优化和 Bug 修复}
      \item {协助设计和实现 UI 组件库}
      \item {参与产品需求分析和技术方案讨论}
    \end{cvitems}
  }

% iOS 技术栈部分
\cvsection{iOS 技术栈}

\cvtag{Swift}
\cvtag{SwiftUI}
\cvtag{Objective-C}
\cvtag{UIKit}
\cvtag{SwiftData}
\cvtag{Core Data}
\cvtag{CloudKit}

\divider

\cvtag{MVVM}
\cvtag{Combine}
\cvtag{RxSwift}
\cvtag{Alamofire}
\cvtag{Kingfisher}
\cvtag{SnapKit}
\cvtag{CocoaPods}
\cvtag{Swift Package Manager}

\divider

\cvtag{Face ID}
\cvtag{Touch ID}
\cvtag{Apple Pay}
\cvtag{Push Notifications}
\cvtag{Background Modes}
\cvtag{Core Location}
\cvtag{Core Bluetooth}
\cvtag{HealthKit}

% 已发布应用部分
\cvsection{已发布应用}

\cvachievement{\faMobile}{智理财}{个人理财管理应用}{超过 100 万下载量,App Store 评分 4.8,功能包括预算管理、投资组合追踪和财务分析}

\cvachievement{\faMobile}{健康助手}{健康管理应用}{集成 HealthKit 和 Core Motion,支持运动追踪、健康监测和数据可视化}

\cvachievement{\faMobile}{即时通讯}{企业通讯应用}{支持实时消息、音视频通话和文件分享,月活跃用户 50 万+}

\cvachievement{\faMobile}{智能家居}{IoT 控制应用}{使用 Core Bluetooth 连接智能设备,支持场景自动化和远程控制}

% 教育背景部分
\cvsection{教育背景}

\cventry
  {2015年9月 - 2019年6月}
  {计算机科学与技术学士}
  {XX 大学}
  {北京}
  {
    \begin{cvitems}
      \item {GPA: 3.9/4.0,专业排名前 5\%}
      \item {主修课程:移动应用开发、算法设计、数据结构、计算机网络}
      \item {毕业设计:基于 iOS 的校园导航应用,获得优秀毕业设计奖}
      \item {Apple 开发者大学俱乐部成员}
    \end{cvitems}
  }

% 证书与认证部分
\cvsection{证书与认证}

\cvachievement{\faCertificate}{2023年}{Apple 认证 iOS 开发者}{通过 Apple 官方认证考试}

\cvachievement{\faCertificate}{2022年}{Swift 高级开发认证}{Swift.org 官方认证}

\cvachievement{\faTrophy}{2022年}{WWDC 学生奖学金}{参加 Apple 全球开发者大会}

\cvachievement{\faCertificate}{2021年}{AWS 云从业者认证}{了解云服务在移动应用中的应用}

% 开源贡献部分
\cvsection{开源贡献}

\cvproject
  {SwiftUI 组件库}
  {https://github.com/zhangsan/swiftui-components}
  {开发并维护一个包含 50+ 可复用 SwiftUI 组件的开源库,获得 2k+ stars}

\cvproject
  {网络请求框架}
  {https://github.com/zhangsan/network-manager}
  {基于 Combine 的网络请求框架,简化 API 调用和错误处理}

% 技术博客与演讲部分
\cvsection{技术博客与演讲}

\cvachievement{\faBlog}{技术博客}{https://medium.com/@zhangsan}{定期发布 iOS 开发技术文章,累计阅读量 10万+}

\cvachievement{\faUsers}{技术分享}{2022年 iOS Meetup}{分享 "SwiftUI 在大型项目中的实践经验"}

\cvachievement{\faUsers}{内部培训}{2021-2023年}{担任公司内部 iOS 开发培训讲师}

\cvachievement{\faBook}{技术专栏}{InfoQ 中文站}{撰写移动开发最佳实践专栏}

% 语言能力部分
\cvsection{语言能力}

\cvskill{中文}{5}
\divider
\cvskill{英语}{5}
\divider
\cvskill{日语}{3}

% 兴趣爱好部分
\cvsection{兴趣爱好}

\cvtag{产品设计}
\cvtag{用户体验}
\cvtag{开源项目}
\cvtag{技术写作}
\cvtag{摄影}
\cvtag{旅行}

\end{document}