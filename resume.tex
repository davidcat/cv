\documentclass[10pt, a4paper]{awesome-cv}

% 编码设置
\usepackage[UTF8]{ctex}
\usepackage{fontspec}
\usepackage{xeCJK}

% 设置中文字体
\setCJKmainfont{PingFang SC}
\setCJKsansfont{PingFang SC}
\setCJKmonofont{PingFang SC}

% 个人信息配置
\name{张}{三}
\address{北京市朝阳区}
\mobile{138-8888-8888}
\email{zhangsan@email.com}
\github{zhangsan}
\linkedin{zhangsan}
\homepage{zhangsan.dev}

% 简历主题颜色
\colorlet{awesome}{awesome-skyblue}

% 开始文档
\begin{document}

% 标题部分
\makecvheader

% 个人简介部分
\cvsection{个人简介}
\begin{cvparagraph}
具有 5 年软件开发经验的全栈工程师,专注于 Web 应用开发和云原生技术。熟练掌握 React、Node.js 和微服务架构,具有丰富的大型项目开发和团队协作经验。热衷于开源技术,致力于构建高性能、可扩展的软件解决方案。
\end{cvparagraph}

% 工作经历部分
\cvsection{工作经历}

\cventry
  {2022年3月 - 至今}
  {高级前端工程师}
  {ABC 科技公司}
  {北京}
  {
    \begin{cvitems}
      \item {负责公司核心产品的前端架构设计和开发,使用 React + TypeScript 构建企业级应用}
      \item {主导前端性能优化,页面加载速度提升 60\%,用户体验显著改善}
      \item {推进组件库标准化,提高团队开发效率 40\%}
      \item {指导初级工程师,组织技术分享,提升团队整体技术水平}
    \end{cvitems}
  }

\cventry
  {2020年6月 - 2022年2月}
  {全栈工程师}
  {XYZ 互联网公司}
  {上海}
  {
    \begin{cvitems}
      \item {参与电商平台全栈开发,使用 Node.js + Vue.js 构建微服务架构}
      \item {设计并实现用户认证系统,支持百万级用户并发访问}
      \item {优化数据库查询性能,响应时间减少 50\%}
      \item {与产品经理紧密合作,快速迭代产品功能}
    \end{cvitems}
  }

\cventry
  {2019年7月 - 2020年5月}
  {前端开发工程师}
  {DEF 软件公司}
  {杭州}
  {
    \begin{cvitems}
      \item {负责公司官网和管理后台的前端开发工作}
      \item {使用 jQuery + Bootstrap 构建响应式网页}
      \item {参与需求分析和技术方案制定}
    \end{cvitems}
  }

% 技术栈部分
\cvsection{技术栈}

\cvtag{React}
\cvtag{Vue.js}
\cvtag{Angular}
\cvtag{Next.js}
\cvtag{JavaScript (ES6+)}
\cvtag{TypeScript}
\cvtag{HTML5}
\cvtag{CSS3}

\divider

\cvtag{Node.js}
\cvtag{Express.js}
\cvtag{Koa.js}
\cvtag{Python}
\cvtag{Java}
\cvtag{Spring Boot}
\cvtag{MySQL}
\cvtag{PostgreSQL}
\cvtag{MongoDB}
\cvtag{Redis}

\divider

\cvtag{Git}
\cvtag{GitHub}
\cvtag{Docker}
\cvtag{Kubernetes}
\cvtag{Jenkins}
\cvtag{AWS}
\cvtag{阿里云}
\cvtag{腾讯云}

% 教育背景部分
\cvsection{教育背景}

\cventry
  {2015年9月 - 2019年6月}
  {计算机科学与技术学士}
  {XX 大学}
  {北京}
  {
    \begin{cvitems}
      \item {GPA: 3.8/4.0,专业排名前 10\%}
      \item {主修课程:数据结构、算法、操作系统、计算机网络、数据库系统}
      \item {获得校级优秀毕业生称号}
    \end{cvitems}
  }

% 项目经验部分
\cvsection{项目经验}

\cventry
  {2022年8月 - 2023年2月}
  {企业级 CRM 系统}
  {技术负责人}
  {}
  {
    \begin{cvitems}
      \item {负责公司核心产品的前端架构设计和开发,使用 React + TypeScript 构建企业级应用}
      \item {主导前端性能优化,页面加载速度提升 60\%,用户体验显著改善}
      \item {推进组件库标准化,提高团队开发效率 40\%}
      \item {系统支持 10,000+ 并发用户,稳定性达到 99.9\%}
    \end{cvitems}
  }

\cventry
  {2021年3月 - 2021年12月}
  {电商平台微服务改造}
  {全栈开发工程师}
  {}
  {
    \begin{cvitems}
      \item {参与电商平台全栈开发,使用 Node.js + Vue.js 构建微服务架构}
      \item {设计并实现用户认证系统,支持百万级用户并发访问}
      \item {优化数据库查询性能,响应时间减少 50\%}
      \item {系统可用性从 95\% 提升到 99.5\%}
    \end{cvitems}
  }

% 证书奖项部分
\cvsection{证书奖项}

\cvachievement{\faTrophy}{2023年}{公司最佳员工奖}{表彰在技术创新和团队协作中的杰出表现}

\cvachievement{\faCertificate}{2023年}{AWS 认证解决方案架构师}{通过 AWS 官方认证考试}

\cvachievement{\faCertificate}{2022年}{PMP 项目管理专业人士认证}{获得 PMI 认证的项目管理资质}

\cvachievement{\faTrophy}{2021年}{黑客马拉松比赛一等奖}{在 48 小时内完成创新项目开发}

% 语言能力部分
\cvsection{语言能力}

\cvskill{中文}{5}
\divider
\cvskill{英语}{4}
\divider
\cvskill{日语}{2}

\end{document}